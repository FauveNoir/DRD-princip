\section{Objet}

\subsection{Définition}
Le Débat rationnel dirigé (\mainabbr{}) est une méthode de discussion et de réflexion collective normée visant à réduire aux seuls éléments significatifs le cours naturel du raisonnement rationnel et ce, en récusant immédiatement les arguments falacieux et les attitudes non constructives, voire, dans l’idéal, que la discipline de débat fasse en sorte qu’ils n’aient même pas lieu, les règles n’existant que pour rappeler à l’ordre des manquements qui se doivent de rester dans la virtualité.

En d’autres termes, l’objectif du \mainabbr{} est de favoriser un cours fluide de la discussion et de l’expurger de tous les bruits de fond non significatifs possibles puisqu’ils finissent prévisiblement par être récusés et donc ne changent en rien le court du raisonnement rationnel et la conclusion qui en découlera mais font, en revanche, perdre du temps, des ressources matérielles et humaines, de la matière mentale et un effort intellectuel suplémentaire qu’il aurait été profitable d’allouer au traitement d’arguments pertinants ou utiles.

\subsection{Cibles}
De façon générale, sont visés par la censure au sein d’un \mainabbr{}, les principaux arguments fallacieux répertoriés, les attitudes irrationnelles (entre autre, celles faisant appel à l’émotion) et, d’une certaine manière, les hors-sujet. C’est à dire les arguments ou les actions
\begin{SRlist}
	\item qui seront récusées
	et
	\item dont cette récusation était de toute façon prévisible avant même que l’argument ne soit introduit%
\end{SRlist}%
.

Autrement dit, lorsqu’il s’agit de bruits de fond qui n’apportent strictement rien à la réflexion et dont l’effet sur le cours de la discution est rigoureusement nul. On pourrait encore dire que ce sont des arguments qui ne font ni conforter d’avantage une certitude que l’on se fait sur un sujet, ni au contraire en font douter ; en bref, desquelles on ne tire littéralement rien.

D’ailleurs, un bon moyen d’identifier de pareils bruits de fond consiste à comparer l’état d’avancement du raisonnement avant l’introduction de pareils arguments avec l’état suivant immédiatement le traitement de cet argument et, si strictement aucune modification ne peut être enregistrée entre les deux états, alors il s’agit d’un bruit de fond.

\paragraph{}
Sauf cas extrêment exceptionnels, circonscrit à des sujets particuliers, où c’est la validité même de la logique intuitioniste et du principe de non-contradiction qui sont remis en cause, ou où le sujet porte sur la physique quantique, tout \mainabbr{} est censé exclure les principaux sophismes procédant par syllogisme répertoriés.

L’idée sous-jacente en est que, l’invalidité de ceux-ci étant formellement établie, il n’est pas nécéssaire d’en refaire la démonstration à chaque occurence mais plutôt d’en \enquote{factoriser} l’irrecevablité, à charge du fautif de se référer à la démonstration. Si, malgré tout, quelques hippias en herbe parmis l’une des parties décident de remettre en cause le caractère fautif d’un sophisme parmis ceux interdits\footnote{Selon les sujets, l’\xenism{Argumentum ad hominem} peut tantot etre banni ou accepté. Par exemple, certains débat peuvent requérir de s’assurer de l’honnêteté d’un témoin.} ils peuvent téoriquement ouvrir un sous-débat portant donc sur la validité du sophisme en question… qui leur donnera nécéssairement tort. Et pour cette raison, une telle pratique est jugée mauvaise et devra donc être évitée. Un sujet aussi fondamental devra nécéssairement faire l’objet d’une très longue réflexion dédiée car pèse dessus une certitude trop largement raisonnable.

\subsection{Délimitations de la déffinition}
\subsubsection{Délimitation des arguments éxclus}
\paragraph*{}
Le \mainabbr{} n’est évidement pas un procédé qui tend ou à vocation à favoriser une conclusion particulière, d’influencer dans un sens ou de n’accepter que les arguments qui heurtent ou au contraire confortent une sensibilité ou un point de vue donné. Ne sont pas concernés par l’exclusion les arguments qui \enquote{ne plaisent pas} en raison d’une supposée mauvaise ésthétique oratoire ou encore de leur innadéquation avec un idéal. Ne sont concernés que les arguments qui, dès lors qu’ils sont traités, n’aboutissent à aucune modification de l’état du débat, ni en bien, ni en mal, ne font pas plus douter d’une idée que conforter la certitude que l’on peut en avoir et ce quelque soit le parti pour lequel ils plaident. Ne sont concernés encore que les attitudes, arguments et autres actions qui n’ont aucune pertinence, et pour lesquels le débat eu {autant} gagné à ce qu’ils n’y soient pas introduits.

\subparagraph*{}
Sont toute fois acceptés les raisonnements s’étant avérés érronés (et ne sont donc pas éxclus du débat), s’ils ont été comis de bonne foi\footnote{Il est difficile de déffinir ce qui est et ce qui n’est pas de bonne foi, mais dans le cadre d’un \mainabbr{} il suffit que l’assentiment général éstime selon toute sa subjectivité qu’il s’agit de bonne foi pour la considérée comme telle.}
% Évasif
et surtout que ce soit une erreure originale qui ne soit pas communément connue dans le sujet traité et dont la fausseté n’a donc pas pu être démontrée avant. Il s’agit là d’un authentique enrichissement du sujet puisqu’elles permettront d’allimenter la base d’erreures répertoriées dans lesquelles l’on évitera de retomber dans la suite du débat. C’est donc bien de la création de \exergue{connaissance}, la connaissance de ce qui ne doit pas être tenu pour vrais.

Il est, toutes fois, toujours possible de rediscuter de ce qui a été considéré comme étant une erreur à condition d’y apporter un élement nouveau sucéptible de prouver qu’il n’en s’agit pas d’une. Pour être acceptée, il n’est pas nécéssaire que l’introduction de ce nouvel élement aboutisse nécéssairement à autre chose qu’un second rejet de l’assertion ; il suffit simplement que cette action n’ai pas un un résultat \exergue{prévisiblement} irrecevable.

\subsubsection{Limites de l’utilité}
\paragraph{}
Enfin, il est nécéssaire de rappeler que le \mainabbr{} ne garranti pas que la conclusion qui en émergera sera nécéssairement la meilleure pouvant exister ou seulement valide logiquement car ce n’est pas un mécanisme de vérification systématique, éttant donné que la récusation des erreures est toujours à l’innitiative d’individus potentiellement faïbles.

Il n’empèche pas ainsi les parties d’un débat de commettre toutes deux la même érreure et de ne pas s’en appercevoir. En revanche, il se contante de soustraire rapidement les érreures les plus évidentes, et, somme toute, concoure à réduire les risques d’érreures dans le raisonnement et dans la conclusion mais ne les anihile pas.

\subparagraph{}
Le déroulement d’un \mainabbr{} ne peut donc être brandit comme preuve de la validité formelle de la conclusion qui en découlerait mais peut témoigner d’une bonne volonté. Il ne consiste pas plus qu’en un procédé d’optimisaton de l’allocation des ressources intellectuelles et d’évaluation de la pértinence d’une discussion.

\subsection{Chanp d’application}
\epigraph{Plus une discussion en ligne dure longtemps, plus la probabilité d’y trouver une comparaison impliquant les nazis ou Adolf \bsc{Hitler} s’approche de 1.}{\fullcite{godwinDeclaration}}
Enfin, le \mainabbr{} ne peut être organisé qu’autour de sujets factuels et objectifs, et cherche a faire émerger d’apprès un processus rationnel une réalité objective sur laquelle on peut s’appuyer. Ainsi est-il évidement insensé de discuter des gouts ou des choix personnels, par exemple.

\paragraph*{}
C’est pourquoi le \mainabbr{} est bien adapté à deux cas distincts ayant cette exigence sus-citée en partage :

\begin{description}
  \item[Situation \enquote{adulte}] Le cas qui concorde le plus avec les objectifs et l’esprit général du \mainabbr{}, est celui où l’ensemble des parties est réputé de bonne foi et, même si chacune en a une opinion pré-établie, elle n’en nourrit pas, à priori, de parti pris émmotionnel ou interessé mais cherche à la confronter au débat contradictoire afin d’obtenir, sinon la solution parfaite, tout du moins, une meilleure solution, plus affinée. Celà correspond très bien à une concertation d’ingénieurs, ou un débat scientifique ou technique, par exemple. Quiconque aurait renoncé à la raison ne devra pas utiliser le \mainabbr{}.
Dans cette situation, les parties ne considèrent leur opinion que comme l’état d’avancement de leur réflection sur le sujet et non comme une certitude dogmatique ou un horizon indépassable. Elle admettent qu’elle ne \exergue{connaissent} pas la réponse à la question débatue mais n’en ont qu’une idée, tout au plus approchante. Aussi le \mainabbr{} leur permet-il justement de faire progrésser leur quêtte d’amélioration.

  \item[Situation \enquote{enfantine}] L’autre cas est celui ou l’on suppose que les parties ont un interêt à manipuler les faits ou que, sans être nécéssairement de mauvaise foi, sont trop émotionnellement impliquées pour admettre la fausseté éventuelle de leur opinion, malgré leur contradiction avec les faits.

  Le \mainabbr{} agit alors comme un mécanisme de controle invalidant les assertions fallacieuses et contraint à l’usage de la rationnalité en pénalisant ceux qui s’en écartent.

  Toute fois, dans la plus part des \exergue{situations enfantines}, l’on peut légitimement s’interroger sur la pertinance à organiser un \mainabbr{} ou toute autre forme de discution car les circonstances y sont d’emblés peut propices à ce que cette discution soit constructive. \enquote{Utiliser des arguments rationnels avec une personne ayant renoncé à la raison c’est comme administrer des médicaments à un mort}\nocite{thomasPain1776AmercianCrisis}, conformément à l’adage. Au delà même du \mainabbr{}, un tel cas, mené à l’extrème, signifierait que les participants abusés par les sophisme ou en abusant eux même aient besoin d’une éducation ou que devrait leur être retirée la résponsabilité de prendre des décisions.
\end{description}


\paragraph{}
L’idéal étant que chaque partie tende à respecter d’elle même les règles auquelles elle s’est proposée de s’astreindre ; chacune pouvant, en cas de violation, rappeler à l’ordre l’autre qui devra s’y plier cordialement (ayant admis que les reproches sont valides au sens des règles du \mainabbr{}). Mais dans un cas de discipline de débat trop discipée, peut intevenir un arbitre neutre auquel seront soumis les objections d’arguments et qui se chargera de les gérer ou d’éguilloner le débat.

Si, malgré tout, les règles du \mainabbr{} sont trop souvent violées, il est nécéssaire d’admettre qu’au moins l’une des parties, le plus souvent celle à l’origine des dites violations, est de mauvaise foi ou qu’elle n’est pas capable de présenter une argumentation pertinante et valide (ce qui revient au même) et qu’il est temps que la discussion prenne fin car devenue stérille et qu’elle ne répondra pas aux critères d’objectivité recherchés par un \mainabbr{}. Un éxeple précis d’un tel cas de figure est, par exemple, illustré par le \enquote{Point de Godwin}.


\section{Définitions}
\epigraph{Entre

Ce que je pense,

Ce que je veux dire,

Ce que je crois dire,

Ce que je dis,

Ce que vous avez envie d’entendre,

Ce que vous entendez,

Ce que vous comprenez…

Il y a dix possibilités qu’on ait des difficultés à communiquer.

Mais essayons quand même…
}{\fullcite{ESRA}}
Dans la suite certains mots seront employés selon une acception particulière.

\begin{description}
  \item[Séssion de \mainabbr{}] Discution cherchant à suivre les règles du \mainabbr{} délimitée dans le temps et circonscrite en nombre de participant.
  \item[Débat] Ensemble de discutions et de raisonements autour d’un sujet qui ont lieux de façon formelle ou informelle à l’extérieur de la session de débat.
  \item[Bruit de fond] Désignation commune à l’ensemble des attitudes non constructives dont la plus part est vouée à être récusée. Un bruit de fond est une action d’un participant qui n’ajoute strictement rien à la réflexion. En particulier, qui ne permet même pas de mêttre en évidence une nouvelle connaissance, qu’elle porte sur la validité ou la faussetée d’une assertion.

    \begin{description}
      \item[Sophisme] Argument falacieux. Argument apparement valide logiquement mais en réalité fautifs. La récusation systématique des sophisme est l’un des principaux objectif du \mainabbr{}.
    \end{description}

  \item[Récusation] Action d’exclure un argument reconnu comme bruit de fond, d’empécher d’en tenir compte pour la suite de la session.
  \item[Objection] Dans les cas où la session de \mainabbr{} est présidée par un arbitre, l’objection est une requette lui étant adréssée pour prononcer la récusation d’un argument.
  \item[Élement significatif] 
  \item[Arbitre] Entité pouvant être composée d’un seul individu ou d’un comité neutre vis à vis du sujet et chargée de veiller au protocole du \mainabbr{}.
  \item[Partie] Entité pouvant être composée d’un seul individu ou d’un groupe éméttant des arguments et pouvant avoir une opinion.
  \item[Participant] Individu prenant part au débat et appartenant à une partie, il peut en être le seul représentant ou l’un des membres multiples.
  \item[Parti pris] Attitude conservatrice d’attachement affectif et irrationnel à une opinion. Refus d’admettre qu’elle puisse être fausse.
  \item[Point de non-retour] État de la discution où il n’est plus possible de la poursuivre de façon constructive.
\end{description}

\paragraph{}
En outre, dans les exemples de discutions seront utilisés des noms de personnages dont les innitiales sont généralemet fonction de l’ordre de prise de parole. Les noms utilisés augureront une personnalité stéréotypée ou une intention. D’autres noms peuvent apparraitre mais souvent, les noms partagent la même innitiales sont ceux de participant appartenant à la même partie.

\begin{description}
  \item[\A] Participant prenant toujours la parole en premier.
  \item[\B] Participant répondant toujours à \A, le plus souvent il est en opposition avec elle.
  \item[\C] Un troisième participant ayant un point de vue autant différent d’\A que de \B.
  \item[\Sophist] Participant utilisant des arguement fallacieux en connaissance de cause. Il est de mauvaise foi et tente de faire parvenir le débat à la concluision qui lui est la plus favorable, et non forcément celle qui est la plus \exergue{juste}.
  \item[\Troll] Participant dont l’objectif est de ruiner la discution, il s’agit d’un troll.
\end{description}

\section{Protocole}
\subsection{Innitialisation}
Une session de débat rationnel dirrigé peut aussi bien avoir lieux par la réunion physique des parties dans un endroit convenu ou encore se faire à distance au moyen de lettres ou mieux, d’un système de communication éllectronique, particulièrement l’Usenet, une liste de diffusion de courriel, ou la messagerie instantannée via l’IRC ou Jabber. À noter qu’un débat sur le long terme peut très bien méler toutes ces formes de communication. Dans les fait, la nature du moyen de communication n’a pas d’importance du point de vue du débat puisque tout argument ou assertion, quelque soit le tunel qu’il ai emprunté ou la forme sous laquelle il est introduit, est sensée être soumis à l’examein critique de la raison avant que sa validité ne soit aprouvée.

Toute fois, pour des raisons logistique, puisque c’est cet aspect sur lequel porte le \mainabbr{}, la forme et les modalités de communication que prendront les échanges devra dont l’idéal être convenue au préalable. Par ailleurs, il est à noter que la communication différée, particulièrement sur l’Usenet, les listes de diffusions, ou tout autre forme présente l’incommensurable avantage de favoriser les raisonnements rigoureux en ce qu’ils permettent à chaque participant de prendre le recul suffisant nécéssaire au résonnement, à la consultation des sources avancées et au peaufinage des arguments.

\paragraph{}
Ayant pris en compte ces considération mais avant que ne commence le débat à proprement parler, il est encore souhaitable de :

\begin{enumerate}
  \item Déffinir éxplicitement l’objectif recherché par les parties afin que l’on ne s’en écarte pas ;

  \item Se mettre d’accord sur les acceptions des mots-clés sensés être utilisés ;

  \begin{enumerate}
    \item Si certains mots ou concepts firent déjà l’objet d’une controverse de nomenclature, il faudrait dans l’idéal s’y référer et adopter les termes que le débat terminologique a préconisé lors d’un précédant \mainabbr{},
  \end{enumerate}

  \item Déffinir les autorités communément reconnues ;

  \item Si, certains sophismes comme l’\xenism{Argumentum ad hominem} sont autorisés, en déffinir les modalités de validité ;

  \item Se référer au compte rendu d’un précédent \mainabbr{} sur le sujet affin d’en retenir les éreures répertoriées à ne pas commettre de nouveau ;

  \item Décider de la mise en palce d’un arbitrage.

\end{enumerate}

\subsubsection{Détail des prés-requis}
\paragraph{Déffinition de l’objectif}
La difficulté à déffinir un objectif est que l’on risque de se limiter beaucoup trop alors que le débat, s’il est sensé enrichir, devrait permettre d’abborder des sujets annexes. Là encore, il appartient aux parties de déffinir si leur objectif est d’avoir un objectif unique ou si dans l’ensemble de la session il peuvent aborder différents aspects d’un sujet complexe. Certains sujets sont par nature transversaux. Mais dans tous les cas, la déffinition préalable de l’objectif permet de ne pas virer vers les sujets connexes mais hors-sujets et dont les uns ne souhaitent pas discuter.

Souvent, le défaut de précision d’objectif empèche de se rendre compte assez tôt des divergences et donc d’agir en conséquence.

\paragraph{}
Dans l’exemple suivant, \A et \B sont associés dans une entreprise de jardinage et discutent des choix à prendre dans l’identité graphique de celle-ci :

\begin{quote}
  \begin{drama}{0cm}{0cm}{Alice}

    \Aspeaks Je voudrais un logotype mauve pour notre entreprise de jardinage parceque j’aime le mauve.

    \Bspeaks Pourtant, dans notre domaine, le vert permet une meilleure visibilité et donc une meilleure apréhension de notre activité.

    \Aspeaks Oui mais j’aime la couleure mauve.

    \Bspeaks Entendons-nous : Allons nous choisir une couleur parceque nous l’\enquote{aimenons} ou parceque c’est celle qui sera la plus favorable à la prospérité de notre entreprise ?

  \end{drama}
\end{quote}

Probablement qu’\A perd de vue les objectifs qui, pour \B, se résument à la contrainte suivante \enquote{Quels choix nous permettront de rentabiliser le plus possible notre activité commerciale ?}. Préciser dès le déppart que toute décision ou tout argument avancé se doit de répondre à cette dessein, pourrait éviter les hors-sujets, et pour chaque participant, mieux cerner les enjeux.

\paragraph{Déffinitions des concepts-clés et mots-clés}
Il est primmordiale de se mettre d’accord sur les acceptions des mots et concepts-pivots qui seront discutés, faute de quoi les participants penseront parler de la même chose alors qu’il s’agit de sujet différents. L’autre risque est qu’à force de surcharger un mot de notions telles qu’on ne sait plus éxactement à laquelle est fait allusion à chaque occurence de ce mot, le débat devient vite vide de sens.

\begin{quote}
  \begin{drama}{0cm}{0cm}{Alice}

    \Aspeaks Je ne suis pas d’accord avec ce dernier argumeent. Je n’accepte pas que les femmes aient les droits que vous dites.

    \Bspeaks Pourtant, vous aviez donné votre accord tantôt pour les accorder à tous les Hommes.

    \Aspeaks Oui j’entendais par là \autonym{homme} au sens de \xenism{vīr}, l’individu masculin, et non \autonym{Homme} sens de \xenism{hommo}, l’individu de l’espèce humaine.

  \end{drama}
\end{quote}

Cet exemple est certes carricatural mais d’autres cas sont plus sournois. Ainsi, en langue française, le mot \autonym{politique}, par exemple, désigne autant la science de la gestion de la cité que le phénomène social de lutte pour le pouvoir. Si les participants se rassemblent au motif qu’ils seraient désireux de parler de \autonym{politique}, encore faudrait-il être sûr que tous souhaitent parler de la gestion de la cité ou de la lutte pour le pouvoir.

\subparagraph{Anticipation des querelles de nomencalture}
Si une précédente séssion de \mainabbr{} donna lieux une querelle de nomenclature qu’il finit par résoudre et, pour peut que les parties en présence aprouvent cette résolution, le mieux encore serait de récupérer l’apport de cette séssion en reprenant au compte de la session actuelle les progrès du compte rendu de la précédente. Celà évitera, de \enquote{ré-inventer la roue carrée}.

\paragraph{Caractère sophistique de certains arguments}
Certains arguments, particulièrement ceux ayant trait à la personne plutôt qu’aux idées, sont fallacieux dans les cas où la corrélation entre les qualités réelles ou supposées d’une personnalité et le sujet traité n’existe pas. En tête de ceux-ci, l’\xenism{Argumentum ad hominem} qui se décline en \xenism{ad verecundiam}, \xenism{ad crumenam}, \xenism{ad lazarum}, mais aussi \xenism{Reductio ad Hitlerum}.

L’attaque personnelle a une validité rigoureusement nulle dans tous les cas où le sujet traité ne porte pas sur la personne faisant l’objet de l’attaque personnelle.

\subparagraph{}
Sauf que, bien entendu, le sujet peut être précisement une personne donnée auquel cas, évidement, l’\xenism{Argumentum ad hominem} est pleinement légitime. D’autre part, si, à un moment donné un participant ou intervenant est interrogé en qualité de témoins, il ne serait pas infondé de juger sa fiabilité, particulièrement son accuité perceptive mais aussi sa crédibilité.

De la même façon, l’\xenism{Argumentum ad pulchrum} est valide si le sujet porte précisèment sur le jugement ésthétique d’un sujet, entre graphistes par exemple. L’\xenism{Argumentum ad populum} est valide lorsqu’il s’agira de discuter d’une chose ayant la propriété de devoir être adaptée à une majorité de personnes.

Mais de façon générale, les arguments portant sur l’appel aux émotions, les déductions érronées, la manipulation de contenus, la confusion entre cause et effet, et la plus part de ceux attaquant la personnalité qui n’est pas l’objet de la discution, ne peuvent jamais être valides.

\subparagraph{}
Celà étant, certaines branches de la mathématique et de la logique peuvent éventuellement avoir pour objet la remise en cause des raisonnements du sens commun, ainsi, au contraire, le raisonnement par l’asbsurde qui est une pratique absolument valide deviendra ainterdite si le sujet porte sur la logique intuitionniste. Auquel cas, il serait prudent de s’accorder dessus.

\paragraph{Déffinition des autorités reconnues} Pour être valides ou considérés comme tels par les parties, un argument d’autorité se doit d’impliquer une autorité reconnue justement comme telle par tout le monde. C’est pourquoi l’on considérera que, par défaut, aucune autorité n’est valide. Le meilleur moyen d’éviter alors que l’une des parties n’ai recourt à une autorité récusée par les autres est encore de se mettre dès le déppart d’accord sur les autorités admises.

\subparagraph{}
Cepadant, il peut vite devenir laborieux de citer toutes les références acceptées et faire une liste exaustive est souvent impossible ou même innutile. Les autorités reconnues par tous dans certains cas sont implicitement connues. Par exemple, dans un débat de théologie impliquant des chrétiens, il est innutile de prendre la peine de citer la Bible comme autorité valide. Tous les participants en ont à peut près l’intuition.

\subparagraph{}
De même, certaines autorités au sein d’un mouvement quoique reconnues par une faction du dit mouvement, leur crédit est toujours sujet à caution. Selon les factions en présence, il sera le plus souvent évident de reconnaitre les autorités que ces dernières récuseront forcément. Par exemple, dans un débat opposant des catholiques et des protestants, soit deux factions du groupe chértiens, un certain Martin \bsc{Luther} n’est évidement pas un dénominateur commun.

\subparagraph{}
Enfin, les références sur l’admiscibilité des quelles planent les doutes les plus sucéptibles sont les autorités médianes, celles fleurtant avec l’hétérodoxie. Par exemple, dans une discution entre communistes, alors que l’autorité de Karl \bsc{Marx} est presque souvent évidente, il faudrait prendre la peine de s’accorder sur la recevabilité par tous de Léon \bsc{Trotski} avant de l’utiliser comme autorité.

\paragraph{Récupération du compte rendu d’une précédente séssion} Un \mainabbr{} est dans l’idéal sensé se cloturer par un compte rendu, souvent enregistré, consignant les progrés éffectués sur un sujet. Il contient comme mentionné plus haut, les résolutions de querelles terminologiques s’il y’en eu mais il peut aussi consigner les différents arguments originaux s’étant avérés invalides. Le mieux, serait encore de repartir de se point afin d’éviter de refaire ce qui a déjà été fait.

\paragraph{Arbitrage} L’arbitrage n’est vraiment nécéssaire qu’en cas de discipline de débat trop délitée ou de participants mal formés. Dans une \enquote{situation adulte}, les individus en présence ne recherchent que la meilleure solution possible et ne rechigneront pas à se corriger eux même s’ils sont rappelés à l’ordre par leurs pairs. Mais rien n’empèche malgré tout un tel rassemblement de ce doter d’un arbitrage dont le rôle sera d’évaluer la validité du moindre argument tandis que les participant vaqueront à leur argumentation.

En fait, un arbittre n’est vraiment nécéssaire que s’il y’a une suspicipon de conflit d’interêt chez au moins l’un des participants ou qu’un autre soit d’humeure trollesque. Dans pareils cas, l’idéal étant que le choix se porte sur un individu réputé indifférent à l’issus du débat et n’ayant aucun interêt manifeste.

Dans tous les cas, il est préférable que la mise en palce d’un arbitrage se décide au début. Un arbittre peut toute fois arriver en cours de séssion, si les intervenants jugent que la discipation devient trop importante.

\subsection{Cloture}

Ayant conclu qu’elle étaient parvenue à la fin du débat, les parties doivent encore, pour celà, vérifier qu’aucun sous-débat n’est en suspent et que toutes les controverses de nomenclatures ont été traitées. Seulement alors peut être prononcée la cloture du débat.

\paragraph{}
Une session de \mainabbr{} n’est pas cloturé lorsqu’elle prend fin pour des raisons logitiques comme la nécéssité pratique pour les participants de se retirer après un temps imparti. En renvenche, la session est cloturée lorsque l’ensemble des parties abouti à une \exergue{conclusion}. Une conclusion pouvant prendre deux principaux aspects
\begin{SRlist}
  \item Soit les parties ont abbouti à une solution satisfaisante ;
  \item Soit l’une des parties éstime qu’aucun accord ne pourra être trouvé et décide de se retirer.
\end{SRlist}
Dans ce dernier cas la cloture n’a lieux que pour la partie s’étant retirée et non pour toutes, s’il y’en avait au moins trois.

Dans le cas contraire, pourra s’organiser la rédaction du compte rendu.

\subsubsection{Ellaboration du compte rendu}
Un débat est un activité gourmande en ressources qui a demandé du temps et de la patience à ses participants. Le mieux encore est qu’à la cloture d’une session de \mainabbr{}, l’on consigne, par écrit idéalement, l’ensemble des enseignements qui en sont issus.

En particulier, il est interessant d’en retenir :
\begin{itemize}
   \item La conclusion ;
   \item Le raisonnement qui a permis d’y aboutir ;
   \item Éventuellement, les idées interessantes connexes révélées au court du débat mais qui étaient alors hors-sujet ;
   \item Les querelles de nomenclatures soulevées et leur éventuelles solutions ;
   \item Les pistes de réflexion s’étant finalement avérées fausses.
\end{itemize}

L’idée étant que l’ensemble du débat entourant une question et que les prochaines séssions de \mainabbr{} pourront hériter des travaux déjà éffectués, afin d’éviter de \enquote{ré-inventer la roue carrée}.
\subsection{Déroulement}
Après l’innitialisation, les échanges commencent. Dans le cas où toutes les interventions des participants sont synchrone, comme l’ors d’une rencontre physique, il sera absolument nécéssaire que chaque personne s’exprime seule à la foi pour éviter les cacophonies. Le participant voulant réagir devra le faire savoir par un signe quelconque convenu (une main levée) et, à charge du maitre de débat de donner son tour à ceux qui ont exprimer le désir de prendre la parole, par ordre de requette. À chaqu’un de prendre une note sur la partie du discours d’autrui à laquelle il voulait réagir.

\paragraph{}
Toute fois, certains évenements, au court de la conversation, peuvent intervenir auquel cas le \mainabbr{} propose des \exergue{mesure} qui preinent effet immédiatement.

Ainsi, les objections en cas de bruit de fond, d’identification de sous-débat, ou de même de querelle de nomenclature auront lieux immédiatement, comme dans l’exemple suivant :
\begin{quote}
  \begin{drama}{0cm}{0cm}{Alice}

    \Aspeaks       $1+1=2$.

    \Bspeaks       Vous n’êtes pas mathématicienne par conséquent tout ce que vous dîtes sur le sujet est nécéssairement faux. S’en suit que $1+1\neq2$, et…

    \Aspeaks       Objection : \xenism{Arguementum ad hominem}.

    \Arbitrespeaks Objection valide : \B. Votre assertion et par conséquent tout ce qui en découle ne sont pas admis. veuillez acréditer votre raisonnement par un autre moyen.

  \end{drama}
\end{quote}

\subsubsection{Particularité des échanges scripturaux}
Si le débat adopte une forme écrite (courrier postal, courriel, clavardage), le grand avantage est que les minutes sont consignées du procédés même. Il n’est nul besoin de greffier. En outre, la forme écrite présente l’incommensurable avantage de fournir à chacun les moyens de lire et d’apréhender les explication à son aise, sans devoir faire répeter un humain. Celà donne le temps de l’annalyse et du traitement de chaque intervention, et la vérification des sources invoquées.

Cepandant, il faudra absolument éviter l’\xenism{Argumentum ad orthographraphiam} qui est éxtrèmement désobligeant, concourt à l’agcement et aux tensions parmis les participants. La session de discution ne doit pas se transformer en immense feuile de correction. Évidement, si l’un des participant se trouve trop dérrangé par une dysorthographie, il devra demander cordialement et en aparthée du débat à l’auteur des fautes de veiller à sa correction orthotipographique. En aucun cas ce sujet ne devra être abordé dans la sphère commune du débat, auquel cas il se vérra objecté comme tout autre sophisme.

\subsubsection{Rôle de l’arbitre}
Un arbittre est sensé être neutre, son rôle se contonne à veiller à la bonne marche du prtocole. Il n’est pas sensé émettre d’opnion mais se contente d’évaluer la validité de celle émises par les participants impliqués. Il peut toute-fois faire remarquer une incohérence et forcer l’une des parties à mieux s’éxpliquer ou à renforcer son argumentation.

C’est pourquoi, l’arbittre est souvent amené à récuser certains arguments, soit spontanément s’il constate de lui-même un sophisme, ou un hors-sujet, soit à la demande d’une des parties.
 Son rôle est aussi d’ouvrir les sous-débats ou de proposer de dépasser les querrelles de nomenclatures, là encore, soit soit de sa propre innitiative, soit à la demande d’une partie.


\section{Traitement des objections}
%TODO Érreures récurentes dans un sujet
\subsection{Sous-débat}
\subsubsection{État des lieux}
Un porblème, pour être résolu, peut nécéssité au préalable de résoudre d’autre problèmes qui sont eux-même des problèmes à part-entière, donc, avec leur propres problèmes-fils. Celà finit par escisser un arbre de raisonnement avec de multiples branches et sous-branches. L’entrée dans une branche, peut déboucher sur des problèmes nécéssitant une réflexion à par entière, parfois laborieuse. Pourtant, il n’est pas possible de revenir à la branche parente sans que celle-ci soit traitée. Parfois même, comme dans un labyrinthe, certaines voies sont sans issus, et il faut revenir à la voie principale pour en explorer d’autres.

\subsubsection{Mesures}
Il est souvent utile de tenir un organigramme de l’arbres du raisonnement afin de garder à l’esprit l’objectif global et les objectifs locaux. Lorsqu’une question à part entière apparait, le sous-débat est déclaré par l’arbittre et il faut l’avoir traiter avant de revenir au sujet principal.

\subsection{Récusation des sophismes}
\epigraph{Quand vous discutez avec un adversaire de droite, traitez le de fasciste, le temps qu’il se défende, il ne pourra pas développer d’autres arguments}{Attribué à \bsc{Staline} par les millieurs fasciste mais jamais authentifiée. Peut importe, cette citation me plait.}
\subsubsection{État des lieux}
Les sophismes sont des arguments fautifs mais en raison de leur apparente validité peuvent induire en erreure. Évidement, tout rhéteur les connais et est cappable d’en démontrer la fausseté. Simplement, cette fausté là a été déjà duement démontrée, documentée à foison. Pour donc éviter de ré-éxprimer la même chose, à chaque occurence d’un sophisme, il sera communément admis que ceux-ci ne seront pas accéptés, sans qu’il y’a besoin d’en refaire la démonstration.

Cela permet, somme toute, d’économiser du temps.

\subsubsection{Mesures}
Les sophismes peuvent être immédiatement objectés et leur auteur doit accepter la récusation qui en sera faite, faute de quoi il sortirait du cadre du \mainabbr{}.
\begin{quote}
  \begin{drama}{0cm}{0cm}{Alice}

    \Aspeaks       $1+1=2$.

    \Bspeaks       Vous n’êtes pas mathématicienne par conséquent tout ce que vous dîtes sur le sujet est nécéssairement faux. S’en suit que $1+1\neq2$, et…

    \Aspeaks       Objection : \xenism{Arguementum ad hominem}.

    \Arbitrespeaks Objection valide : \B, votre assertion et par conséquent tout ce qui en découle ne sont pas admis. veuillez acréditer votre raisonnement par un autre moyen.

  \end{drama}
\end{quote}

\subsection{Retour à un point antérieur}
\subsubsection{État des lieux}
D’une certaine façon, le retour au point antérieur peut s’apparenter avec l’\xenism{Argumentum ad nauseam} et peut être traité comme les autres sophismes. Toute fois, puisque son effet sur le flux du raisonnement est différent des autres sophismes (il implique un retour à un point antérieur et fait fit des avancées réalisées), il est traité différement.

\begin{quote}
  \begin{drama}{0cm}{0cm}{Alice}

    \Aspeaks       Les systèmes carcéraux rigoureux sont un gage de dissuasion.

    \Bspeaks       Pourtant, statistiquement, les sociétés ayant les systèmes carcéraux les plus dures sont celle où la criminalité demeure hausse. Si comme vous le dite la prison était dissuasive, la criminalité devrait baisser dans de pareils endroits.

    \Aspeaks       Il n’empèche que la prison est dissuasive.

  \end{drama}
\end{quote}

\subsubsection{Mesure}
Il suffit de faire constater le retour et en suite de revenir au point précédent immédiatement ce retour là.
\begin{quote}
  \begin{drama}{0cm}{0cm}{Alice}

    \Aspeaks       Il n’empèche que la prison est dissuasive.

    \Bspeaks       Objection : Retour à un point antérieur, reffut de traiter l’argument opposé.

    \Arbitrespeaks Objection valide : \A, \B démontre que votre assertion est fausse, à moins que vous ne démontrez en quoi la sienne est fausse ou que vous proposiez une autre assertion, celle de \B sera considérée comme la meilleure connue.

  \end{drama}
\end{quote}

Ici, \B confond l’argument d’\A mais celle-ci fait comme si \B n’avait rien dit et répète la phrase qu’elle avait prononcé juste avant. L’État du débat n’a strictement pas progréssé et le fait qu’elle répette a tue-tête son assertion ne la rend pas plus juste. Elle se doit soit de traiter l’argument de \B soit d’admettre que sur ce point son contre-argument est valide et de proposer un autre argument.

\subsection{Dépassement des querelles de nomenclature}
\epigraph{N’a de sens que ce à quoi l’on daigne en accorder.}{Fauve. \work{Les rugissements}.}
\subsubsection{État des lieux}
Du fait de la polysémie inhérente au languages naturelles, faisant que deux concepts proches mais distincts soient désignés par le même mots, certaines discutions s’enlisent dans une dispute sur la déffinition d’un mot plutôt que sur le concept auquel il renvoit. Il se peut aussi que l’une des parties concentre son argumentation sur un concept relativement proche ou réputé de la même classe ou ensemble dont celui dont il s’agit réellement alors que malgré la parenté entre ces deux concept les deux demeurent distincts essentiellement.

Or, souvent le concept parent de celui traité par le débat en cours peut ne pas présenter d’interêt dans la discution et devenir de ce fait hors sujet.

\paragraph{}
Le cas typique d’une telle querelle de nomenclature porte souvent sur la désignation mouvements idéologiques pouvant compter un sous-ensemble dense de courrants, parfois même rivaux. Mais le fait que tous ces courrants partagent la même dénomination du mouvement dont ils font partie induit à quelques équivoques.

Pour illustrer une telle controverse, prenons un exemple où est débatue la nature du mouvement dénomé \enquote{féminisme}.



\begin{quote}
  \begin{drama}{0cm}{0cm}{Alice}

    \Aspeaks Le féminisme est une misandrie.

    \Bspeaks Pas du tout, le féminisme ne cherche que le bien des femmes non le mal des hommes.

    \Aspeaks Alors comment expliquez vous que le \citetitle{solanas1967scum} est un pamphlet féministe pronant l’extermination des individus de sexe masculin ?

    \Bspeaks Il existe plusieurs courrants féministes, or je ne m’exprime pas au nom de tous ceux-ci. Tout ce que je défend ici, quelque soit la désignation sous laquelle vous l’appeller, c’est le fait que les hommes et les femmes soient égaux.

    \Aspeaks Donc, vous voulez l’égalité, non le féminisme.

    \Bspeaks Si, mon attitude égalitariste se réclame du fémisme.

  \end{drama}
\end{quote}

\paragraph{}
Dans cet exemple là, \B est partisant d’un idéal précis et circonscrit qui peut s’exprimer prosaïquement de la sorte \enquote{L’égalité entre les personnes, abstraction faite de leur sexe.}, par ailleurs, il dénome cet idéal \enquote{féminisme}, quoique dans le fond, la dénomination lui importe peut, l’éssentiel étant pour lui que son idéal soit appliqué.
C’est pourquoi l’idée qu’il déffend ici ne peut être tenue pour contable des conséquences des autres idées n’ayant avec elle en partage que le nom.

Pour \A, la même dénomination \enquote{féminisme} renvoit à un autre idéal que \B ne défend pas. Pourtant, ils se retrouvent à discuter justement de la réprobation que tous deux émettent sur le féminisme au sens que connait \A et se retrouvent donc à être virtuellement en désacord sur un sujet où il sont pourtant d’accord.

En suite, l’évaluation des idées digresse vers un débat portant srictement sur les dénomination et non plus sur le bien-fondé des idées et mesures à préconiser.


\subsubsection{Mesure}
Dans le fond, le débat terminologique, en vértue de l’hypothèse Sapir—Whorf n’est pas peut pertinant, simplement il est hors sujet pendant le débat sémantique et vient parasiter les raisonnement au sujet des faits ce qui alourdit et encombre le débat. Il ne faut pas que la discution se transforme en compétition de \enquote{Qui connait la bonne déffinnition}.

\paragraph{}
\subparagraph{Explicitation}
Une première solution simple consiste à préciser à chaque occurence l’acception selon laquelle est utilisée un mot, ainsi, dans l’exemple précédent, on pourra parler de \enquote{Féminisme selon \A} et de \enquote{Féminisme selon \B}.

\subparagraph{Pronoms}
Simplement, pour éviter les risques de confusion, et la lourdeure de l’expression, une autre solution préconisée par le \mainabbr{} consiste alors à donner des noms neutres à chacun des concepts que chaque partie veillera à utiliser par la suite de la session de débat. Il est par exemple possible de privillégier l’alphabet grec pour la formation des noms de concepts. Ainsi, l’un des concept pourra prendre le nom d’\enquote{Alpha} et l’autre de \enquote{Beta}.

Les pronoms alphabétiques grecs pourront être utilisés comme des noms propres \enquote{Le principe ou la personne Alpha}. Et cette nomenclature accepte évidement les déclinaisons, particulièrement dans les langues agglutinantes. En français, il s’agirat généralement de suffixiation. En l’occurence, pour l’exemple précédent sur le féminisme, le \enquote{féminisme selon \A} pourra simplement être désigné par la suite d’\enquote{Alphaïsme}, et celui de \B de \enquote{Betaïsme}.

\paragraph{}
Évidement, dès que le débat sémantique aura pris fin ou sera suspendu, il sera alors temps de se consacrer au débat terminologique à part entière si les parties estiment qu’il est pertinant. Deux cas de figure s’offriront alors.
\begin{itemize}
  \item Soit il s’agissait d’un mot simple que la consultation d’un dictionnaire de référence suffit à en disciper l’ignorance.

  \item Soit, comme dans l’exemple ci-dessus, le mot controversé est véritablement ambigüe et il s’agira alors d’éllaborer les désignations qui prévaudront pour désigner les concepts.
\end{itemize}

\subparagraph{}
Dans l’exemple précédent, une conclusion acceptable est de considérer que le terme \enquote{fénisme} englobe des acceptions diverses et que seul, sans précision, il est trop vague. Ainsi, l’on concluera qu’il est absurde de parler \exergue{du} féminisme comme un ensemble monolitique alors qu’il y’a \exergue{des} féminismes.


\subsection{Hors sujets}
\subsubsection{État des lieux}
En fait, le hors-sujet peut-être vu comme les sophismes puisque les deux n’apportent strictement rien au débat. Toute fois, ils sont moins agressifs. Par ailleurs, ils interviennent toujours par défaut de précision de l’objet du débat, d’où l’interêt d’y remédier en innitialisation.

\subsubsection{Mesures}
Les hors-sujets sont traités de la même façon que les sophismes, une objection est émise avec rappel de l’objet du débat et confrontation avec le hors-sujet.

\begin{quote}
  \begin{drama}{0cm}{0cm}{Alice}

    \Aspeaks       Je voudrais un logotype mauve pour notre entreprise de jardinage parceque j’aime le mauve.

    \Bspeaks       Pourtant, dans notre domaine, le vert permet une meilleure visibilité et donc une meilleure apréhension de notre activité.

    \Aspeaks       Oui mais j’aime la couleure mauve.

    \Bspeaks       Objection : Hors sujet. La poroposition d’\A répond à la question \enquote{Qu’est-ce qui nous plait le plus ?} alors que le sujet du débat était éxplicitement \enquote{Quelle mesures seront les plus favorable à la prospérité de notre entreprise ?}.

    \Arbitrespeaks Objection valide. \A veuillez contenir vos émotions et aider à répondre à la question.

  \end{drama}
\end{quote}
