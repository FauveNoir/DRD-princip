\section{Objet}

\subsection{Déffinition}
Le Débat rationnel dirrigé (\mainabbr{}) est une méthode de discussion et de réflexion collective normée visant à réduire aux seuls éléments significatifs le cours naturel du raisonnement rationnel et ce en récusant immédiatement les arguments falacieux et les attitudes non-constructives voir, dans l’idéal, que la discipline de débat fasse en sorte qu’ils n’aient même pas lieu, les règles n’éxistant que pour rappeler à l’ordre des manquements qui se doivent de rester dans la virtualité.

En d’autres termes, l’objectif du \mainabbr{} est de favoriser un cours fluide de la discussion et de l’expurger de tous les bruits de fond non-significatifs possibles puisqu’ils finissent prévisiblement par être récusés et donc ne changent en rien le court du raisonnement rationnel et la conclusion qui en découlera mais font, en revenche, perdre du temps, des ressources matérielles et humaines, de la matière mentale et un effort intellectuel suplémentaire qu’il aurait été profitable d’allouer au traitement d’arguments pertinants ou utiles.

\subsection{Cibles}
De façon générale, sont visés par la censure au sein d’un \mainabbr{}, les principaux arguments fallacieux répértoriés, les attitudes irrationnelles (entre autre, celles faisant appel à l’émotion) et, d’une certaine manière, les hors-sujet. C’est à dire les arguments ou les actions
\begin{SRlist}
	\item qui seront récusées
	et
	\item dont cette récusation était de toute façon prévisible avant même que l’argument ne soit introduit%
\end{SRlist}%
.

Autrement dit, lorsqu’il s’agit de bruits de fond qui n’apportent strictement rien à la réflexion et dont l’effet sur le cours de la discution est rigoureusement nul. On pourrait encore dire que ce sont des arguments qui ne font ni conforter d’avantage une certitude que l’on se fait sur un sujet, ni au contraire en font douter ; en bref, desquelles on ne tire littéralement rien.

D’ailleurs, un bon moyen d’identifier de pareils bruits de fond consiste à comparer l’état d’avancement du raisonnement avant l’introduction de pareils arguments avec l’état suivant immédiatement le traitement de cet argument et, si strictement aucune modification ne peut être enregistrée entre les deux états, alors il s’agit d’un bruit de fond.

\paragraph{}
Sauf cas extrèmement exceptionnels, circonscrit à des sujets particuliers, où c’est la validité même de la logique intuitioniste et du principe de non-contradiction qui sont remis en cause, ou où le sujet porte sur la physique quantique, tout \mainabbr{} est censé exclure les principaux sophismes procédant par syllogisme répertoriés.
% sophismes de logique.


L’idée sous-jascente en est que, l’invalidité de ceux-ci étant formellement établie, il n’est pas nécéssaire d’en refaire la démonstration à chaque occurence mais plutôt d’en \enquote{factoriser} l’irrecevablité, à charge du fautif de se référer à la démonstration. Si, malgré tout, quelques hippias en herbe parmis l’une des parties décident de remettre en cause le caractère fautif d’un sophisme parmis ceux interdits\footnote{Selon les sujets, l’\xenism{Argumentum ad hominem} peut tantot etre banni ou accepté. Par exemple, certains débat peuvent requérir de s’assurer de l’honnêteté d’un témoin.} ils peuvent téoriquement ouvrir un sous-débat portant donc sur la validité du sophisme en question… qui leur donnera nécéssairement tort. Et pour cette raison, une telle pratique est jugée mauvaise et devra donc être évitée. Un sujet aussi fondamental devra nécéssairement faire l’objet d’une très longue réflexion dédiée car pèse dessus une certitude trop largement raisonnable.

\subsection{Délimitations de la déffinition}
\subsubsection{Délimitation des arguments éxclus}
\paragraph*{}
Le \mainabbr{} n’est évidement pas un procédé qui tend ou à vocation à favoriser une conclusion particulière, d’influencer dans un sens ou de n’accepter que les arguments qui heurtent ou au contraire confortent une sensibilité ou un point de vue donné. Ne sont pas concernés par l’exclusion les arguments qui \enquote{ne plaisent pas} en raison d’une supposée mauvaise ésthétique oratoire ou encore de leur innadéquation avec un idéal. Ne sont concernés que les arguments qui, dès lors qu’ils sont traités, n’aboutissent à aucune modification de l’état du débat, ni en bien, ni en mal, ne font pas plus douter d’une idée que conforter la certitude que l’on peut en avoir et ce quelque soit le parti pour lequel ils plaident. Ne sont concernés encore que les attitudes, arguments et autres actions qui n’ont aucune pertinence, et pour lesquels le débat eu {autant} gagné à ce qu’ils n’y soient pas introduits.

\subparagraph*{}
Sont toute fois acceptés les raisonnements s’étant avérés érronés (et ne sont donc pas éxclus du débat), s’ils ont été comis de bonne foi\footnote{Il est difficile de déffinir ce qui est et ce qui n’est pas de bonne foi, mais dans le cadre d’un \mainabbr{} il suffit que l’assentiment général éstime selon toute sa subjectivité qu’il s’agit de bonne foi pour la considérée comme telle.}
% Évasif
et surtout que ce soit une erreure originale qui ne soit pas communément connue dans le sujet traité et dont la fausseté n’a donc pas pu être démontrée avant. Il s’agit là d’un authentique enrichissement du sujet puisqu’elles permettront d’allimenter la base d’erreures répertoriées dans lesquelles l’on évitera de retomber dans la suite du débat. C’est donc bien de la création de \exergue{connaissance}, la connaissance de ce qui ne doit pas être tenu pour vrais.

Il est, toutes fois, toujours possible de rediscuter de ce qui a été considéré comme étant une erreur à condition d’y apporter un élement nouveau sucéptible de prouver qu’il n’en s’agit pas d’une. Pour être acceptée, il n’est pas nécéssaire que l’introduction de ce nouvel élement aboutisse nécéssairement à autre chose qu’un second rejet de l’assertion ; il suffit simplement que cette action n’ai pas un un résultat \exergue{prévisiblement} irrecevable.

\subsubsection{Limites de l’utilité}
\paragraph{}
Enfin, il est nécéssaire de rappeler que le \mainabbr{} ne garranti pas que la conclusion qui en émergera sera nécéssairement la meilleure pouvant exister ou seulement valide logiquement car ce n’est pas un mécanisme de vérification systématique, éttant donné que la récusation des erreures est toujours à l’innitiative d’individus potentiellement faïbles.

Il n’empèche pas ainsi les parties d’un débat de commettre toutes deux la même érreure et de ne pas s’en appercevoir. En revanche, il se contante de soustraire rapidement les érreures les plus évidentes, et, somme toute, concoure à réduire les risques d’érreures dans le raisonnement et dans la conclusion mais ne les anihile pas.

\subparagraph{}
Le déroulement d’un \mainabbr{} ne peut donc être brandit comme preuve de la validité formelle de la conclusion qui en découlerait mais peut témoigner d’une bonne volonté. Il ne consiste pas plus qu’en un procédé d’optimisaton de l’allocation des ressources intellectuelles et d’évaluation de la pértinence d’une discussion.

\subsection{Chant d’application}
\epigraph{Plus une discussion en ligne dure longtemps, plus la probabilité d’y trouver une comparaison impliquant les nazis ou Adolf \bsc{Hitler} s’approche de 1.}{Mike \bsc{Godwin}. \url{<1991Aug18.215029.19421@eff.org>}. 1991~août~18. \textsc{url :}~\url{nntp:rec.arts.sf-lovers}}
Enfin, le \mainabbr{} ne peut être organisé qu’autour de sujets factuels et objectifs, et cherche a faire émerger d’apprès un processus rationnel une réalité objective sur laquelle on peut s’appuyer. Ainsi est-il évidement insensé de discuter des gouts ou des choix personnels, par exemple.

\paragraph*{}
C’est pourquoi le \mainabbr{} est bien adapté à deux cas distincts ayant cette exigence sus-citée en partage :

\begin{description}
  \item[Situation \enquote{adulte}] Le cas qui concorde le plus avec les objectifs et l’esprit général du \mainabbr{}, est celui où l’ensemble des parties est réputé de bonne foi et, même si chacune en a une opinion pré-établie, elle n’en nourrit pas, à priori, de parti pris émmotionnel ou interessé mais cherche à la confronter au débat contradictoire afin d’obtenir, sinon la solution parfaite, tout du moins, une meilleure solution, plus affinée. Celà correspond très bien à une concertation d’ingénieurs, ou un débat scientifique ou technique, par exemple. Quiconque aurait renoncé à la raison ne devra pas utiliser le \mainabbr{}.
Dans cette situation, les parties ne considèrent leur opinion que comme l’état d’avancement de leur réflection sur le sujet et non comme une certitude dogmatique ou un horizon indépassable. Elle admettent qu’elle ne \exergue{connaissent} pas la réponse à la question débatue mais n’en ont qu’une idée, tout au plus approchante. Aussi le \mainabbr{} leur permet-il justement de faire progrésser leur quêtte d’amélioration.

  \item[Situation \enquote{enfantine}] L’autre cas est celui ou l’on suppose que les parties ont un interêt à manipuler les faits ou que, sans être nécéssairement de mauvaise foi, sont trop émotionnellement impliquées pour admettre la fausseté éventuelle de leur opinion, malgré leur contradiction avec les faits.

  Le \mainabbr{} agit alors comme un mécanisme de controle invalidant les assertions fallacieuses et contraint à l’usage de la rationnalité en pénalisant ceux qui s’en écartent.

  Toute fois, dans la plus part des \exergue{situations enfantines}, l’on peut légitimement s’interroger sur la pertinance à organiser un \mainabbr{} ou toute autre forme de discution car les circonstances y sont d’emblés peut propices à ce que cette discution soit constructive. \enquote{Utiliser des arguments rationnels avec une personne ayant renoncé à la raison c’est comme pratiquer de la médecine sur un mort}, conformément à l’adage. Au delà même du \mainabbr{}, un tel cas, mené à l’extrème, signifierait que les participants abusés par les sophisme ou en abusant eux même aient besoin d’une éducation ou que devrait leur être retirée la résponsabilité de prendre des décisions.
\end{description}


\paragraph{}
L’idéal étant que chaque partie tende à respecter d’elle même les règles auquelles elle s’est proposée de s’astreindre ; chacune pouvant, en cas de violation, rappeler à l’ordre l’autre qui devra s’y plier cordialement (ayant admis que les reproches sont valides au sens des règles du \mainabbr{}). Mais dans un cas de discipline de débat trop discipée, peut intevenir un arbitre neutre auquel seront soumis les objections d’arguments et qui se chargera de les gérer ou d’éguilloner le débat.

Si, malgré tout, les règles du \mainabbr{} sont trop souvent violées, il est nécéssaire d’admettre qu’au moins l’une des parties, le plus souvent celle à l’origine des dites violations, est de mauvaise foi ou qu’elle n’est pas capable de présenter une argumentation pertinante et valide (ce qui revient au même) et qu’il est temps que la discussion prenne fin car devenue stérille et qu’elle ne répondra pas aux critères d’objectivité recherchés par un \mainabbr{}. Un éxeple précis d’un tel cas de figure est, par exemple, illustré par le \enquote{Point de Godwin}.


\section{Déffinitions}
Dans la suite certains mots seront employés selon une acception particulière.

\begin{description}
  \item[Séssion de \mainabbr{}] Discution cherchant à suivre les règles du \mainabbr{} délimitée dans le temps et circonscrite en nombre de participant.
  \item[Débat] Ensemble de discutions et de raisonements autour d’un sujet qui ont lieux de façon formelle ou informelle à l’extérieur de la session de débat.
  \item[Bruit de fond] Désignation commune à l’ensemble des attitudes non constructives dont la plus part est vouée à être récusée. Un bruit de fond est une action d’un participant qui n’ajoute strictement rien à la réflexion. En particulier, qui ne permet même pas de mêttre en évidence une nouvelle connaissance, qu’elle porte sur la validité ou la faussetée d’une assertion.

    \begin{description}
      \item[Sophisme] Argument falacieux. Argument apparement valide logiquement mais en réalité fautifs. La récusation systématique des sophisme est l’un des principaux objectif du \mainabbr{}.
    \end{description}

  \item[Récusation] Action d’exclure un argument reconnu comme bruit de fond, d’empécher d’en tenir compte pour la suite de la session.
  \item[Objection] Dans les cas où la session de \mainabbr{} est présidée par un arbitre, l’objection est une requette lui étant adréssée pour prononcer la récusation d’un argument.
  \item[Élement significatif] 
  \item[Arbitre] Entité pouvant être composée d’un seul individu ou d’un comité neutre vis à vis du sujet et chargée de veiller au protocole du \mainabbr{}.
  \item[Partie] Entité pouvant être composée d’un seul individu ou d’un groupe éméttant des arguments et pouvant avoir une opinion.
  \item[Parti pris] Attitude conservatrice d’attachement affectif et irrationnel à une opinion. Refus d’admettre qu’elle puisse être fausse.
  \item[Point de non-retour] État de la discution où il n’est plus possible de la poursuivre de façon constructive.
\end{description}

\section{Protocole}
%nécéssité de se mettre d’accord sur l’objectif à atteindre.
%déffinition précise des concepts abordés
%sophisme
%retour à un point précédent
%hors sujet
%sous-débat
%Déffinition préalable des autorités communément reconues.
%controverse de nomenclature — Déffinition des concepts-pivots et mots-clés.
%Érreures récurentes dans un sujet
%Rôle de l’arbittre
\subsection{Dépassement des querelles de nomenclature}
\epigraph{N’a de sens que ce à quoi on en accorde.}{Fauve. \work{Les rugissements}.}
\subsubsection{État des lieux}
Du fait de la polysémie inhérente au languages naturelles, faisant que deux concepts proches mais distincts soient désignés par le même mots, certaines discutions s’enlisent dans une dispute sur la déffinition d’un mot plutôt que sur le concept auquel il renvoit. Il se peut aussi que l’une des parties concentre son argumentation sur un concept relativement proche ou réputé de la même classe ou ensemble dont celui dont il s’agit réellement alors que malgré la parenté entre ces deux concept les deux demeurent distincts essentiellement.

Or, souvent le concept parent de celui traité par le débat en cours peut ne pas présenter d’interêt dans la discution et devenir de ce fait hors sujet.

\paragraph{}
Le cas typique d’une telle querelle de nomenclature porte souvent sur la désignation mouvements idéologiques pouvant compter un sous-ensemble dense de courrants, parfois même rivaux. Mais le fait que tous ces courrants partagent la même dénomination du mouvement dont ils font partie induit à quelques équivoques.

Pour illustrer une telle controverse, prenons un exemple où est débatue la nature du mouvement dénomé \enquote{féminisme}.


\Character[Alice]{Alice}{A}
\Character[Bob]{Bob}{B}

\begin{quote}
  \begin{drama}{0cm}{0cm}{Alice}

    \Aspeaks Le féminisme est une misandrie.

    \Bspeaks Pas du tout, le féminisme ne cherche que le bien des femmes non le mal des hommes.

    \Aspeaks Alors comment expliquez vous que le \work{SCUM Manifesto} est un pamphlet féministe pronant l’extermination des individus de sexe masculin ?

    \Bspeaks Il existe plusieurs courrants féministes, or je ne m’exprime pas au nom de tous ceux-ci. Tout ce que je défend ici, quelque soit la désignation sous laquelle vous l’appeller, c’est le fait que les hommes et les femmes soient égaux.

    \Aspeaks Donc, vous voulez l’égalité, non le féminisme.

    \Bspeaks Si, mon attitude égalitariste se réclame du fémisme.

  \end{drama}
\end{quote}

\paragraph{}
Dans cet exemple là, \B est partisant d’un idéal précis et circonscrit qui peut s’exprimer prosaïquement de la sorte \enquote{L’égalité entre les personnes, abstraction faite de leur sexe.}, par ailleurs, il dénome cet idéal \enquote{féminisme}, quoique dans le fond, la dénomination lui importe peut, l’éssentiel étant pour lui que son idéal soit appliqué.
C’est pourquoi l’idée qu’il déffend ici ne peut être tenue pour contable des conséquences des autres idées n’ayant avec elle en partage que le nom.

Pour \A, la même dénomination \enquote{féminisme} renvoit à un autre idéal que \B ne défend pas. Pourtant, ils se retrouvent à discuter justement de la réprobation que tous deux émettent sur le féminisme au sens que connait \A et se retrouvent donc à être virtuellement en désacord sur un sujet où il sont pourtant d’accord.

En suite, l’évaluation des idées digresse vers un débat portant srictement sur les dénomination et non plus sur le bien-fondé des idées et mesures à préconiser.


\subsubsection{Mesure}
Dans le fond, le débat terminologique, en vértue de l’hypothèse Sapir—Whorf n’est pas peut pertinant, simplement il est hors sujet pendant le débat sémantique et vient parasiter les raisonnement au sujet des faits ce qui alourdit et encombre le débat. Il ne faut pas que la discution se transforme en compétition de \enquote{Qui connait la bonne déffinnition}.

\paragraph{}
\subparagraph{Explicitation}
Une première solution simple consiste à préciser à chaque occurence l’acception selon laquelle est utilisée un mot, ainsi, dans l’exemple précédent, on pourra parler de \enquote{Féminisme selon \A} et de \enquote{Féminisme selon \B}.

\subparagraph{Pronoms}
Simplement, pour éviter les risques de confusion, et la lourdeure de l’expression, une autre solution préconisée par le \mainabbr{} consiste alors à donner des noms neutres à chacun des concepts que chaque partie veillera à utiliser par la suite de la session de débat. Il est par exemple possible de privillégier l’alphabet grec pour la formation des noms de concepts. Ainsi, l’un des concept pourra prendre le nom d’\enquote{Alpha} et l’autre de \enquote{Beta}.

Les pronoms alphabétiques grecs pourront être utilisés comme des noms propres \enquote{Le principe ou la personne Alpha}. Et cette nomenclature accepte évidement les déclinaisons, particulièrement dans les langues agglutinantes. En français, il s’agirat généralement de suffixiation. En l’occurence, pour l’exemple précédent sur le féminisme, le \enquote{féminisme selon \A} pourra simplement être désigné par la suite d’\enquote{Alphaïsme}, et celui de \B de \enquote{Betaïsme}.

\paragraph{}
Évidement, dès que le débat sémantique aura pris fin ou sera suspendu, il sera alors temps de se consacrer au débat terminologique à part entière si les parties estiment qu’il est pertinant. Deux cas de figure s’offriront alors.
\begin{itemize}
  \item Soit il s’agissait d’un mot simple que la consultation d’un dictionnaire de référence suffit à en disciper l’ignorance.

  \item Soit, comme dans l’exemple ci-dessus, le mot controversé est véritablement ambigüe et il s’agira alors d’éllaborer les désignations qui prévaudront pour désigner les concepts.
\end{itemize}

\subparagraph{}
Dans l’exemple précédent, une conclusion acceptable est de considérer que le terme \enquote{fénisme} englobe des acceptions diverses et que seul, sans précision, il est trop vague. Ainsi, l’on concluera qu’il est absurde de parler \exergue{du} féminisme comme un ensemble monolitique alors qu’il y’a \exergue{des} féminismes.




%Bibliographie
%	Point de godwin

